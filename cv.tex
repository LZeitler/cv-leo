% compiler: xelatex cv.tex
% run: xelatex cv.tex; okular cv.pdf

\let\latexnofiles\nofiles
\let\nofiles\relax

\documentclass[margin]{res}
\usepackage[german]{babel}
\usepackage[T1]{fontenc}
% \usepackage{unicode-math}
\usepackage{hyperref}
\usepackage{etaremune}
\usepackage[sc,osf]{mathpazo}

% \hyphenation{Hohenheim}

%\usepackage{helvet} % Default font is the helvetica postscript font
%\usepackage{newcent} % To change the default font to the new century schoolbook postscript font uncomment this line and comment the one above

\setlength{\textwidth}{13cm} % 5.1in (13cm) Text width of the document

% \topmargin=-0.5cm
\oddsidemargin -.6cm  % negative val moves stuff to left
\addtolength{\resumewidth}{.6cm}
\addtolength{\textwidth}{.6cm}
\addtolength{\textheight}{2cm}
% \evensidemargin -1.5cm % does nothing here 
% \textwidth=6.0in
% \itemsep=0in
% \parsep=0in

\begin{document}

%----------------------------------------------------------------------------------------
%	NAME AND ADDRESS SECTION
%----------------------------------------------------------------------------------------

\moveleft.5\hoffset\centerline{\large\bf Leo Zeitler} % Your name at the top

\moveleft\hoffset\vbox{\hrule width\resumewidth height .5pt}\smallskip % Horizontal line after name; adjust line thickness by changing the '1pt'

\moveleft.5\hoffset\centerline{University of Bern} % Your address
\moveleft.5\hoffset\centerline{Institute of Plant Sciences} % Your address
\moveleft.5\hoffset\centerline{Altenbergrain 21} % Your address
\moveleft.5\hoffset\centerline{3013 Bern}
\moveleft.5\hoffset\centerline{Switzerland}
\moveleft.5\hoffset\centerline{\href{mailto:leo.zeitler@gmail.com}{leo.zeitler@gmail.com}}
\moveleft.5\hoffset\centerline{\href{https://twitter.com/leo_zeitler}{twitter: @leo\_zeitler}}
\moveleft.5\hoffset\centerline{\href{https://github.com/LZeitler}{GitHub: LZeitler}}
% \moveleft.5\hoffset\centerline{+41\,(0)\,78\,34\,34\,077}

%-------RESUME STARTS HERE---------------------------------------------------------------

\begin{resume}
%----------------------------------------------------------------------------------------
%	EDUCATION SECTION
%----------------------------------------------------------------------------------------

\section{Education}

\textbf{\textit{PhD Candidate Population Genomics}} \hfill 10/2020--current \\
Institute of Plant Sciences, University of Bern, Switzerland \\
Advisors: Dr.~Kimberly Gilbert \& Prof.~Dr.~Christian Parisod


\textbf{\textit{M.Sc. Crop Sciences}} \hfill 09/2016--03/2019 \\
University of Hohenheim, Stuttgart, Germany \\
Thesis: Loss of genetic diversity in doubled-haploid lines from European maize landraces \\
% Description: Analysis of genomic data of 137 individuals from five European landrace populations and 404 doubled haploid lines derived from these populations to study the potential loss of genetic diversity.\\
Advisors: Prof.~Jeffrey Ross-Ibarra, Dr.~Markus Stetter \& Prof.~Karl Schmid


\textbf{\textit{B.Sc. Agricultural Sciences}} \hfill 04/2013--11/2016 \\
University of Hohenheim, Stuttgart, Germany\\
Thesis: Sequence analysis of putative domestication genes in Amaranth\\
% Description: Sequence analysis of domestication related genes of different crop plants in the \textit{Amaranthus hypochondricus} genome, evaluation of the reciprocal best hit method for identifying orthologous and paralogous genes.\\
Advisors: Prof.~Karl Schmid

{\sl High School/Abitur} \hfill 2003--2011 \\
Johannes Kepler Gymnasium, Stuttgart

%----------------------------------------------------------------------------------------
%	PROFESSIONAL EXPERIENCE SECTION
%----------------------------------------------------------------------------------------
 
\section{Work Experience}

{\sl Research internship} \hfill 04/2019--10/2020 \\
Bomblies Lab, Plant Evolutionary Genetics, Institute for Molecular Plant Biology, Department of Biology, ETH Zurich --- Zurich, Switzerland


{\sl Research internship} \hfill 04/2018--10/2018 \\
Ross-Ibarra Lab, Department of Plant Sciences and Center for Population Biology, University of California Davis --- Davis, CA, USA
% \\

% \begin{itemize}\itemsep -2pt \vspace{-12pt}  % Reduce space between items
% \item analysed genomic data of 137 individuals from five European landrace populations and 404 doubled haploid lines derived from these populations to study the potential loss of genetic diversity.
%   % \vspace{-12pt}
% \end{itemize}


{\sl Student assistant} \hfill 09/2015--03/2018 \\
Institute of Plant Breeding, Seed Science and Population Genetics, University of Hohenheim --- Stuttgart, Germany

% \begin{itemize}\itemsep -2pt \vspace{-12pt}  % Reduce space between items
% \item assisting in various types of experiments 
% \item crossings and screening of amaranths and quinoa
% \item screening of quinoa genbank accessions
  % \vspace{-12pt}
% \end{itemize}


{\sl Internship -- Plant Breeding} \hfill 08/2017--10/2017 \\
Betaseed Inc. (KWS) --- Kimberly, ID, USA
%\\

% \begin{itemize}\itemsep -2pt \vspace{-12pt} 
% \item plant breeding company breeding sugar beet, US-subsidiary company of KWS
% \item Breeders provided insights into practical breeding process and
% taught about US-american markets and GM-specific traits and sugarbeet
% breeding features. 
% \item visited field trials and assisted with various types of tasks (transplanting, harvesting, cleaning, packaging)
  % \vspace{-12pt}
% \end{itemize}

% Reference: Thomas Koeps\ \ $\cdotp$\ \ \href{mailto:Thomas.Koeps@kws.com}{Thomas.Koeps@kws.com}


{\sl Internship -- Plant Breeding} \hfill 06/2014--10/2014 \\
PZO Oberlimpurg  --- Schwäbisch Hall, Germany
%\\

% \begin{itemize}\itemsep -2pt \vspace{-12pt} 
% \item plant breeding company breeding spelt, barley, wheat, durum, oats and soy.
% \item assisted in phenotypic selection and maintenance breeding trials 
% % \item harvested crosses and conducted quality control in laboratory. 
% % \item the breeding team provided insight into the cereal breeding program
%   % \vspace{-12pt}
% \end{itemize} 

%----------------------------------------------------------------------------------------
%	PUBLICATIONS SECTION
%----------------------------------------------------------------------------------------
 
\section{Publications} % Cell Style using Zotero copy to clipboard
% \subsection{Submitted}
% \begin{etaremune}
% \setlength\itemsep{0ex}
% \item \textbf{Zeitler, L.}, Ross-Ibarra, J., and Stetter, M.G. (2020). Selective Loss of Diversity in Doubled-Haploid Lines from European Maize Landraces. G3: Genes, Genomes, Genetics 10, 2497–2506. DOI:\href{https://doi.org/10.1534/g3.120.401196}{10.1534/g3.120.401196}
% \end{etaremune}


\subsection{Published}
\begin{etaremune}
\setlength\itemsep{0ex}
%\begin{itemize} \itemsep -2pt % Reduce space between items

\item Weitz, A.P., Dukic, M., \textbf{Zeitler, L.}, and Bomblies, K. (2021). Male meiotic recombination rate varies with seasonal temperature fluctuations in wild populations of autotetraploid Arabidopsis arenosa. Molecular Ecology 30, 4630–4641. DOI:\href{https://doi.org/10.1111/mec.16084}{10.1111/mec.16084}

\item \textbf{Zeitler, L.}, Ross-Ibarra, J., and Stetter, M.G. (2020). Selective Loss of Diversity in Doubled-Haploid Lines from European Maize Landraces. G3: Genes, Genomes, Genetics 10, 2497-2506. DOI:\href{https://doi.org/10.1534/g3.120.401196}{10.1534/g3.120.401196}

\item Stetter, M.G., \textbf{Zeitler, L.}, Steinhaus, A., Kroener, K., Biljecki, M., and Schmid, K.J. (2016). Crossing Methods and Cultivation Conditions for Rapid Production of Segregating Populations in Three Grain Amaranth Species. Front. Plant Sci. 7. DOI:\href{https://doi.org/10.3389/fpls.2016.00816}{10.3389/fpls.2016.00816}
  
% \end{itemize}
\end{etaremune}


%---------------------------------------------------------------------------------------
%	AWARDS AND SCHOLARSHIPS
%----------------------------------------------------------------------------------------

\section{Grants \& Awards} 
\begin{itemize}\itemsep -2pt %\vspace{-8pt} 
\item KWS Master Scholarship (2016-2018)
\item Herzog Carl Scholarship (2018)
\item Travel grant from Baden-Württembergische Ministerium für Wissenschaft, Forschung und Kunst (2018)
\item 3rd Poster Prize at ELLS Student Conference, Wageningen, 2018
\item Best student research group project of the Agricultural Faculty, 4. Humboldt reloaded-Jahrestagung, 2015
\end{itemize}


%---------------------------------------------------------------------------------------
%	COMPUTER SKILLS SECTION
%----------------------------------------------------------------------------------------

\section{Interest \& Skills } 

{\sl{Genetics}}\\
population genetics (diversity, domestication, adaptation, genome dynamics), quantitative genetics (GWAS, mixed models)
 
{\sl{Bioinformatics}}\\
Statistical analysis (\textsf{R}, \textsc{SAS}), python, population genetics and bioinformatics using \textsf{R} and command line tools (plink, vcftools, BLAST, beagle, GATK, samtools), simulations using SLiM, bash scripting, cluster computing (slurm, LSF), galaxy, \LaTeX, emacs, MacOS, GNU/Linux

{\sl{Wet-Lab}}\\
DNA extraction, plant cultivation

%----------------------------------------------------------------------------------------
%	Languages
%----------------------------------------------------------------------------------------

\section{Languages}
\newlength{\langbox} 
\settowidth{\langbox}{Germang}
\parbox[t]{\langbox}{\sl{German}}\ \ $\cdotp$\ \ \ Mother tongue\\
\parbox[t]{\langbox}{\sl{English}}\ \ $\cdotp$\ \ \ Advanced
% \parbox[t]{\langbox}{\sl{Latin}}\ \ $\cdotp$\ \ \ Basic

%----------------------------------------------------------------------------------------
%	Refs
%----------------------------------------------------------------------------------------

\section{References}

Prof.~Jeffrey Ross-Ibarra\ \ $\cdotp$\ \ \href{mailto:rossibarra@ucdavis.edu}{rossibarra@ucdavis.edu}\\
Dr.~Markus Stetter\ \ $\cdotp$\ \ \href{mailto:m.stetter@uni-koeln.de}{m.stetter@uni-koeln.de}\\
Prof.~Kirsten Bomblies\ \ $\cdotp$\ \ \href{mailto:kirsten.bomblies@biol.ethz.ch}{kirsten.bomblies@biol.ethz.ch}\\
Prof.~Karl Schmid\ \ $\cdotp$\ \ \href{mailto:karl.schmid@uni-hohenheim.de}{karl.schmid@uni-hohenheim.de}\\


%----------------------------------------------------------------------------------------

\end{resume}
\end{document}